\documentclass{../../../../docs/tex/report}
\usepackage{setspace} % Setting line spacing
\usepackage{ulem} % Underline
\usepackage{caption} % Captioning figures
\usepackage{subcaption} % Subfigures
\usepackage{geometry} % Page layout
\usepackage{multicol} % Columned pages
\usepackage{array,etoolbox}
\usepackage{fancyhdr}
\usepackage{enumitem}
\usepackage[toc,page]{appendix}

% Page layout (margins, size, line spacing)
\geometry{letterpaper, left=1in, right=1in, bottom=1in, top=1in}
\setstretch{1.5}

% Headers
\pagestyle{fancy}
\lhead{PeaPod for XYZ - Requirements}
\rhead{PeaPod Technologies Inc.}

% Metric counter, referencing commands
\newcounter{metricnumber}
\setcounter{metricnumber}{1}
\newcommand{\metricrow}{M\arabic{metricnumber}}
\newcommand{\mlabel}[1]{\addtocounter{metricnumber}{-1}\refstepcounter{metricnumber}\label{#1}\addtocounter{metricnumber}{1}}
\newcommand{\mref}[1]{\hyperref[#1]{M\ref{#1}}}

\begin{document}

\begin{titlepage}
    \begin{center}
        \vspace*{1.2cm}

        \textbf{\large{PeaPod for XYZ - Requirements}}

        \vspace{0.5cm}

        Outlining the Implementation-Specific Requirements for a PeaPod XYZ\\

        \textit{Extends: \textbf{PeaPod - Requirements}}

        \vfill
        \input{../../../../docs/tex/documentation/Namecard.tex}
        \vspace{1.25cm}

        Revision 0.1\\
        PeaPod Technologies Inc.\\
        March 31st, 2024

    \end{center}
\end{titlepage}

\thispagestyle{plain}

\tableofcontents
\newpage

\section{Introduction}
\label{sec:intro}

\subsection{Purpose}
\label{sec:purpose}

The purpose of this document is to outline both the category requirements (Section \ref{sec:requirements}) for an implementation of the PeaPod framework (See \textit{PeaPod - Requirements}) that XYZ and the scoped requirements (Section \ref{sec:scope}) for the design being proposed by PeaPod Technologies Inc.: \textbf{PeaPod for XYZ}.

\subsection{Design Paradigm}
\label{sec:structure}

\input{../../../../docs/tex/documentation/DesignParadigm.tex}

\clearpage


\subsection{Scope and Justification}
\label{sec:scope}

\begin{enumerate}[label=SC\arabic*., ref=SC\arabic*]
    \item \label{sc:1} Lorem ipsum dolor sit amet, consectetur adipiscing elit:
    \begin{enumerate}[label=SC3\alph*., ref=SC3\alph*]
        \item \label{sc:1a} Sed auctor, nunc nec ultricies ultricies, nunc nunc ultricies nunc, nec ultricies nunc nunc nec.
    \end{enumerate}
\end{enumerate}

\subsection{Definitions}
\label{sec:definitions}

A number of useful definitions have emerged from the above scoping:
\begin{enumerate}
    \item \textbf{ABC} - Lorem ipsum dolor sit amet, consectetur adipiscing elit.
\end{enumerate}

\clearpage


\section{Framing}
\label{sec:framing}

\subsection{Problem Statement}
\label{sec:opportunity}

Lorem ipsum dolor sit amet, consectetur adipiscing elit. Sed auctor, nunc nec ultricies ultricies, nunc nunc ultricies nunc, nec ultricies nunc nunc nec.

\subsection{Solution Requirements}
\label{sec:requirements}

The following are the overall challenge requirements compiled from A, B, and an excerpt from C:
\begin{enumerate}[label=R\arabic*., ref=R\arabic*]
    \item \label{r:1} \textbf{Must} lorem ipsum dolor sit amet, consectetur adipiscing elit:
    \begin{enumerate}[ref=R1\alph*]
        \item \label{r:1a} \textbf{Should} lorem ipsum dolor sit amet, consectetur adipiscing elit;
    \end{enumerate}
\end{enumerate}

% Change line spacing for the more list-heavy sections
\setstretch{1}
\subsection{Stakeholders and Values}
\label{sec:stakeholders}

\begin{enumerate}[label=S\arabic*., ref=S\arabic*]
    \item \label{s:1} A - Values, etc.
    \item \label{s:2} B - DfX, etc.
\end{enumerate}

\clearpage


\subsection{Problem-Solving Goals}
\label{sec:goals}

% High-Level
\begin{multicols}{2}[]
    \begin{enumerate}[label=HL\arabic*., ref=HL\arabic*]
        \item \label{hl:output} Goal ABC \hfill (\ref{s:1}, \ref{r:1}, \ref{r:1a})
    \end{enumerate}
\end{multicols}

\subsection{Solution Objectives}
\label{sec:objectives}

% Low-Level
\begin{multicols}{2}[]
    \begin{enumerate}[label=LL\arabic*., ref=LL\arabic*]
        \item \label{ll:output_variety} Objective ABC \hfill (\ref{hl:output})
    \end{enumerate}
\end{multicols}

\clearpage


\subsection{Metrics}
\label{sec:metrics}

\begin{tabular}{| @{\makebox[2.4em][c]{\metricrow}} | p{8.7cm} | p{5.9cm} |} 
    \hline
    \multicolumn{1}{| @{\makebox[2.4em][c]{\textbf{\#}}} | l |}{\textbf{Metric}} & \textbf{Units}\\ 
    \hline
    Metric ABC \mlabel{m:constraint} \hfill (\ref{ll:output_variety}) & Yes/No \\
    Metric XYZ \mlabel{m:criteria} \hfill (\ref{ll:output_variety}) & 0 - 100\% \\
    \hline
\end{tabular}

\clearpage


\subsection{Constraints}
\label{sec:constraints}

\begin{tabular}{|l|p{14.35cm}|}
    \hline
    \textbf{Metric} & \textbf{Constraint \hfill Justification} \\
    \hline
    \mref{m:constraint} & Yes \hfill (\ref{s:1})\\
    \hline
\end{tabular}

\subsection{Criteria}
\label{sec:criteria}

\begin{tabular}{|l|p{14.35cm}|}
    \hline
    \textbf{Metric} & \textbf{Criteria \hfill Justification} \\
    \hline
    \mref{m:criteria} & Should Maximize \hfill (\ref{r:1a}) \\
    \hline
\end{tabular}

% Refer to Appendix \ref{sec:assessment} for prototype verification Assessment Criteria (categories, weights, etc.).

\newpage

\section{Reference Designs}

% -------- TEMPLATE --------
% Introduction - Project goal, scope, differences from this project
% Graphics - Design drawings/photos, etc.
% Analysis - Rank the design across each of our metrics
    % TODO: Metrics might be too much, maybe just qualitative analysis based on LLOs?

\subsection{Reference Design XYZ}

Lorem ipsum dolor sit amet, consectetur adipiscing elit. Sed auctor, nunc nec ultricies ultricies, nunc nunc ultricies nunc, nec ultricies nunc nunc nec.

% \newpage

% % References
% \bibliographystyle{IEEEtran}
% \bibliography{references}

\end{document}